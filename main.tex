%! TeX program = lualatex

\documentclass{fit-teorsem}

%-------------------------------------------------------------------------------
%                 Fill in seminar information
%-------------------------------------------------------------------------------
\lecturername{Ondřej Kvapil}
\lectureremail{kvapiond@fit.cvut.cz}
\papertitle{The Complexity of Interaction}
\paperauthors{Stéphane Gimenez, Georg Moser}
\paperlink{https://dl.acm.org/doi/10.1145/2914770.2837646}

%-------------------------------------------------------------------------------
%                 Use custom packages
%-------------------------------------------------------------------------------
\usepackage{enumitem}
\usepackage{amsmath}
\usepackage{amsfonts}
\usepackage{tikz}
\usetikzlibrary{arrows,cd,positioning,shapes,fit}

\tikzset{
	encircle/.style = {draw, circle, inner sep = 0.5mm, color = red},
	dot/.style = {circle, minimum size = 1mm, inner sep = 0mm, draw = black},
	reflexive dot/.style={loop,looseness=17,in=130,out=50},
	reflexive above/.style={->,loop,looseness=7,in=120,out=60},
	reflexive below/.style={->,loop,looseness=7,in=240,out=300},
	reflexive left/.style={->,loop,looseness=7,in=150,out=210},
	reflexive right/.style={->,loop,looseness=7,in=30,out=330}
}

\begin{document}
%-------------------------------------------------------------------------------
%                 Print seminar header
%-------------------------------------------------------------------------------
\maketsheader
%-------------------------------------------------------------------------------
%                 Create your content!
%-------------------------------------------------------------------------------
\thispagestyle{empty}

\section*{Notes}
\begin{itemize}
	\item It may be confusing to see that data constructors pass results on
		principal ports, but functions don't use principal ports for outputs
	\item how are interaction nets better than parallel $\beta$-reduction of
		the lambda calculus?
	\item there's a lot of linear logic stuff, how far into the references
		do we want to go?
	\item we should definitely go over the basics of rewriting systems
		(notably explain confluence, aka the diamond property) and
		probably over a brief reminder of what the LC actually is
		\begin{itemize}
			\item we can use \verb|\faDiamond| from the FontAwesome package
				for the diamond property
		\end{itemize}
	\item is the timed sequential reduction \textit{required} to exhibit
		the diamond property, or is that clear from its definition?
\end{itemize}

\section*{Definitions}

\section*{Theorems}

\end{document}
